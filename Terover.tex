\documentclass[a4paper, 12pt]{article}

\usepackage[T2A]{fontenc}
\usepackage[utf8]{inputenc}
\usepackage[english,russian]{babel}
\usepackage{amsmath, amsfonts, amssymb, amsthm, mathtools, graphicx}

\author{Булинский Андрей Вадимович}
\title{Теория вероятностей. Лекции и Семинары}
\date{\today}
\begin{document}
	\maketitle
	\newpage
	\tableofcontents
	\newpage
	\section{Лекция 1}
	\subsection{Предмет изучения}
	Теория вероятностей изучает закономерности, присущие случайным явлениям. Неслучайные явления будем называть детерминированными. В курсе будем изучать модели случайных экспериментов.\par
	Модель случайных экспериментов подразумевает:
	\begin{enumerate}
		\item Воспроизводимость (контроль основных факторов).
		\item Непредсказуемость исходов.
	\end{enumerate}
	\subsection{Частотная интерпретация вероятностей}
	Основные понятия:\\
	Имеется серия из $N$ повторений эксперимента.\\
	$A$ – явление (событие), которое может произойти.\\
	$N(A)$ – число экспериментов, когда  произошло.\\
	$\nu_N\left(A\right)=\frac{N\left(A\right)}{N}$ – частота события  в серии из  повторений.\\
	\\
	\paragraph{Свойство стабилизации:}
	Пусть $N_1\gg 1$ и $N_2\gg 1$, то $\nu_{N_1}\left(A\right)\approx\nu_{N_2}\left(A\right)$.\\
	$P\left(A\right)$ – вероятность.
	\subsection{Вероятностное пространство}
	Математической моделью случайного эксперимента является вероятностное пространство. Для упрощения задачи используем математический аппарат теории множеств (и теории мер).\par
	Вероятностное пространство состоит из трех множеств $\left(\Omega, F, P\right)$.
	\begin{enumerate}
		\item Непустое множество $\Omega$ (омéга большое) – всевозможные элементарные исходы эксперимента. Пояснение: Элементарные исходы – простейшие, взаимоисключающие исходы.
		\theoremstyle{definition}
		\newtheorem{exmp}{Пример}[section]
		\begin{exmp}
		однократное подбрасывание монеты. Комментарий: пример с монеткой крайне популярен как в русскоязычных, так и в англоязычных пособиях, поэтому будем использовать числовые значения для обозначения исходов эксперимента: $\Omega=\left\{\text{Г},\text{Р}\right\}$, $\Omega=\left\{H,T\right\}$, $\Omega=\left\{0,1\right\}$.\\
		Здесь введем понятия мощности множества – числа элементов конечного множества. Обозначение: $\left|\Omega\right|=\#\Omega=2$
		\end{exmp}
		
	\end{enumerate}
\end{document}
