\documentclass[a4paper, 12pt]{article}

\usepackage[T2A]{fontenc}
\usepackage[utf8]{inputenc}
\usepackage[english, russian]{babel}

\usepackage{amsmath, amsfonts, amssymb, amsthm, mathtools, graphicx}

\author{Булинский Андрей Вадимович}
\title{Теория вероятностей. Лекции и Семинары}
\date{\today}
\begin{document}
\maketitle
\pagenumbering{gobble}
\newpage
\tableofcontents
\pagenumbering{arabic}
\newpage
\section{Лекция }
\subsection{Предмет изучения}
Теория вероятностей изучает закономерности, присущие случайным явлениям. Неслучайные явления будем называть детерминированными. В курсе будем изучать модели случайных экспериментов.\\
Модель случайных экспериментов подразумевает:
\begin{enumerate}
\item Воспроизводимость (контроль основных факторов)
\item Непредсказуемость исходов
\end{enumerate}
\subsection{Частотная интерпретация вероятностей}
Основные понятия:\\
Имеется серия из $N$ потоврений эксперимента . $A$ – явление (событие), которое может произойти. $N(A)$ — число экспериментов, когда $A$ произошло. $\nu_N\left(A\right)=\frac{N\left(A\right)}{N}$ — частота событий $A$ в серии из $N$ повторений.\\
\textit{Свойство стабилизации:\\}
Пусть $N_1\gg 1$ и $N_2\gg 1$, тогда $\nu_{N_1}\left(A\right)\approx\nu_{N_2}\left(A\right)$.\\
$P\left(A\right)$ — вероятность.
\subsection{Вероятностное пространство}
Математической моделью случайного эксперимента является вероятностное пространство. Для упрощения задачи используем математический аппарат теории множеств (и теории мер)\\
Вероятностное пространство состоит из трех множеств $\left(\Omega, F, P\right)$.
\begin{enumerate}
\item Непустое множество $\Omega$ (омéга большое) – всевозможные элементарные исходы эксперимента. Пояснение: Элементарные исходы – простейшие, взаимоисключающие исходы. \\
\\
\begin{theorem}
\textit{Пример 1:} однократное подбрасывание монеты. Комментарий: пример с монеткой крайне популярен как в русскоязычных, так и в англоязычных пособиях, поэтому будем использовать числовые значения для обозначения исходов эксперимента: $\Omega=\left\{\text{Г, Р}\right\}, \Omega=\{H,T\}, \Omega=\{0,1\}$.\\
Здесь введем понятия мощности множества – числа элементов конечного множества. Обозначение: $\left|\Omega\right|=\#\Omega=2$\\
\end{theorem}
\\
\begin{theorem}
\textit{Пример 2:} Эксперимент: \textit{n}-кратное подбрасывание монеты, $\left(n\in N\right)$.\\
Введем понятие элементарных исходов $\omega$ (омéга малое).\\
$\omega\in\Omega, \omega=\left(k_1,\dddot{},k_n\right)$ где $k_j\in\left\{0,1\right\};j=\overline{1,N}$. Где $\omega$ – двоичное \textit{N}-разрядное слово. Добавим, что мощность $\Omega$ в данном случае $\left|\Omega\right|=2^N$. В теории вероятностей $\Omega$ — пространство элементарных исходов. 
\end{theorem}
\end{enumerate}
\end{document}
